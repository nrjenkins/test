\documentclass[letterpaper,10pt]{article}
\usepackage[utf8]{inputenc}
\usepackage[margin=1in]{geometry}
\usepackage[colorlinks=true,urlcolor=Blue,bookmarks=true]{hyperref}
\usepackage{titlesec}
\usepackage{longtable}
\usepackage[dvipsnames]{xcolor}
\usepackage{graphicx}
\usepackage{fancyhdr}
\usepackage{xifthen}
\usepackage{titling}
\usepackage{bookmark}
\usepackage{etaremune}
\usepackage{graphbox}


% Bibliography Setup %
\usepackage[authordate, natbib, isbn=false, url=false, doi=true, backend=biber]{biblatex-chicago} 
\bibliography{/Users/nick/Documents/Research/References/BibTeX/biblatex.bib}


%Defining the entry command:
\newcommand{\entry}[4]{
\ifthenelse{\isempty{#3}}
{\slimentry{#1}{#2}}{

\begin{minipage}[t]{.135\textwidth}
\hfill \textsc{#1}
\end{minipage}
\hfill\vline\hfill
\begin{minipage}[t]{.82\textwidth}
{\bf#2}\\\textit{#3}. \footnotesize{#4}
\end{minipage}\\
\vspace{.25cm}
}}

\newcommand{\slimentry}[2]{

\begin{minipage}[t]{.13\textwidth}
\hfill \textsc{#1}
\end{minipage}
\hfill\vline\hfill
\begin{minipage}[t]{.82\textwidth}
#2
\end{minipage}\\
}


%Footnote Symbol%
\renewcommand{\thefootnote}{$\star$} 


% Entry for Longer Left Side
\newcommand{\longentry}[4]{

\begin{minipage}[t]{.20\textwidth}
\hfill \textsc{#1}
\end{minipage}
\hfill\vline\hfill
\begin{minipage}[t]{.77\textwidth}
{#2}
\end{minipage}\\
\vspace{.25cm}
}


% Entry for Publications
\newcommand{\pub}[3]{

\begin{minipage}[t]{.95\textwidth}
{#1}. \footnotesize{#2} \medskip
\end{minipage}\\
\begin{minipage}[t]{\linewidth}
\hspace{0.5cm} \textit{\small{#3}}
\end{minipage}
}



% Macros for Schools
\newcommand{\ucr}{University of California, Riverside}
\newcommand{\csulb}{California State University, Long Beach}
\newcommand{\csuf}{California State University, Fullerton}

%Macros for people's names
\newcommand{\tgb}{\href{http://coglanglab.com}{Tom Bever}}

%Link images
\newcommand{\pdf}{\includegraphics[align=c, height=.65\baselineskip]{PDF.jpg}}
%\newcommand{\yt}{\includegraphics[height=.75\baselineskip]{yt.png}}
\newcommand{\gh}{\includegraphics[align=c, height=.65\baselineskip]{github.png}}
\newcommand{\cert}{\includegraphics[align=c, height=1\baselineskip]{certificate2.png}}
\newcommand{\grades}{\includegraphics[align=c, height=1\baselineskip]{grades.png}}

\newcommand{\vcenteredinclude}[1]{\begingroup
\setbox0=\hbox{\includegraphics[height=.65\baselineskip]{#1}}%
\parbox{\wd}{\box0}\endgroup}

%Section spacing and format:
\titleformat{\section}{\Large\scshape\raggedright}{}{0em}{}[\titlerule]
\titlespacing{\section}{0pt}{3pt}{7pt}
\titleformat{\subsection}{\large\sc\centering}{}{0em}{\underline}%[\rule{3cm}{.2pt}]
\titlespacing{\subsection}{0pt}{7pt}{7pt}

\setlength{\parindent}{0in}
\setlength{\parindent}{0in}

% Header and Footer Format
\pagestyle{fancy}
\fancyhf{}
\renewcommand{\headrulewidth}{0pt}
\chead{\footnotesize \textsc{Nicholas R. Jenkins -- Curriculum Vitae}}
\rfoot{\footnotesize \textsc{Page \thepage}}
\lfoot{\footnotesize \today}

% Header for the First Page
\fancypagestyle{firststyle}
{
	\fancyhf{}
	\fancyfoot[l]{\footnotesize \today}
	\fancyfoot[r]{\footnotesize \textsc{Page \thepage}}
}


\begin{document}

\thispagestyle{firststyle}

\begin{center}
\par{{\Huge \textsc{Nicholas R. Jenkins}}\par}
\end{center}


\section{Basic Info}

\vspace{.25cm}

\begin{minipage}[t]{.5\linewidth}

\begin{tabular}{rp{.75\linewidth}}
    \textsc{Website:} &\href{https://nicholasrjenkins.science}{nicholasrjenkins.science} \\
    \textsc{Email:} & \href{mailto:nicholas.jenkins@email.ucr.edu}{nicholas.jenkins@email.ucr.edu}
\end{tabular}
\end{minipage}
\begin{minipage}[t]{.5\linewidth}
\begin{tabular}{rl}
	\textsc{Google Scholar:} & \href{https://scholar.google.com/citations?hl=en&view_op=list_works&gmla=AJsN-F4YjeAdLDKT6kbtDT4HzOpmrZ0YDVa6iwfZVHoEtijVG8isg02qW3PBEUcGznxez3amN5uSzcXmNUHD1e4P_BWi3tu55w&user=FPXYTX0AAAAJ}{Nicholas R. Jenkins} \\
	\textsc{Github:} & \href{http://github.com/nrjenkins}{nrjenkins}
\end{tabular}
\end{minipage}

\vspace{.35cm}

% Research Interests %


\vspace{.25cm}

\section{Academic Positions}

\entry{Spring 2020}
			{Master of Public Policy Capstone Project Consultant}
			{\ucr}
			{I was hired by the School of Public Policy as a quantative methods consultant for Master of Public Policy students' capstone projects.}


% Education %
\section{Education}

\medskip

\entry{\emph{In Progress}}
			{Doctor of Philosophy, Political Science}
			{\ucr}
			{Dissertation Committee: Kevin Esterling (Chair)}
			
\entry{2020}
			{Master of Arts, Political Science}
			{\ucr}
			{Specialized in American politics.}
			
\entry{2017}
			{Master of Arts, Economics}
			{\csulb}
			{Specialized in international finance, economic development, and labor economics.}
			
\entry{2015}
			{Bachelor of Arts, Business Administration}
			{\csuf}
			{Major in Accounting; minor in Economics.}

\vspace{.25cm}


% Awards %
\section{Awards, Fellowships, and Grants}

\medskip

\entry{2017}
			{Chancellor's Distinguished Fellowship}
			{\ucr}
			{}

\entry{2017}
			{College of Liberal Arts Distinguished Graduate in Economics}
			{\csulb}
			{}

\entry{2017}
			{1st Place, 29th Annual Student Research Competition - \$100}
			{\csulb}
			{}

\entry{2016}
			{Graduate Research Fellowship - \$9,000}
			{\csulb}
			{The Graduate Research Fellowship is awarded by CSULB faculty to students who show potential for success in scholarly and creative activity and an interest in advanced study. Graduate candidates must be nominated by a faculty member and there will only be one fellowship recipient per college.}

\entry{2016}
			{Simonson Economics Department Scholarship - \$500}
			{\csulb}
			{}

\entry{2015}
			{Formuzis-Pickersgill-Hunt Student Paper Award - \$500}
			{\csuf}
			{This scholarship was established by the Formuzis, Pickersgill \& Hunt, Inc. Economic Consultants. Submitted papers, written per guidelines, will be judged by the Economics Department Student Affairs Committee for the undergraduate awards.}
			{}

\vspace{.25cm}

% Research %
\section{Research}

For current research projects, see my website \href{https://nicholasrjenkins.science/#publications}{nicholasrjenkins.science}.

\medskip

\textbf{\textsc{Peer-Reviewed Publications}}

\begin{etaremune}
	\item \pub{\fullcite{jenkins2020c}}{}{}
	\item \pub{\fullcite{meyer2019}}{}{}
\end{etaremune}

%\nocite{meyer_what_nodate}
%\nocite{jenkins_determinants_2015}
%\printbibliography[heading=none]

\vspace{.25cm}


% Conference Presentations %
\section{Conference Presentations and Workshops}

\medskip

\entry{2020}
			{Teaching for Student Learning (with \href{https://www.michelangelo.mx}{Michelangelo Landgrave})}
			{Presented during the \href{https://gsrc.ucr.edu/aftergrad/tcw}{Teaching Development Day} at the University of California, Riverside}

\entry{2020}
			{Transparency or Deception? How Rejecting PAC Contributions Affects Contribution Patterns}
			{Accepted (Conference Canceled) at the Association for Public Policy Analysis \& Management Regional Student Conference, Riverside, CA}
			{}

\entry{2020}
			{Debtors and Democracy: The Effect of Student Loans on Political Participation (with Alex Ross)}
			{Accepted (Conference Canceled) at the Western Political Science Association Annual Meeting, Los Angeles, CA}
			{}

\entry{2020}
			{Disruption in State Legislatures: Term Limits and Representational Linkages}
			{Accepted (Conference Canceled) at the Western Political Science Association Annual Meeting, Los Angeles, CA}
			{}
			
\entry{2017}
			{M.A. in Economics Math Review \href{https://nrjenkins.github.io/nrjenkins.github.io/files/tutorials/math_tutorial.pdf}{\pdf}}
			{\csulb}
			{I created, and hosted, the math review session for the incoming cohort of graduate students in economics at \csulb.}
			
\entry{2016}
			{Instrumental Variables Regression \href{https://github.com/nrjenkins/workshops/blob/master/iv_regression/IV_presentation.pdf}{\pdf} \href{https://github.com/nrjenkins/workshops/tree/master/iv_regression}{\gh}}
			{Statistics Workshop, University of California, Irvine}
			
\entry{2016}
			{Intermediate Micro/Macroeconomics Math Review Session}
			{\csulb}

\vspace{.25cm}


% Teaching Experience %
\section{Relevant Teaching Experience}

\medskip

\textbf{\textsc{Teaching Assistant}\footnote{Scores calculated using the following item: \textit{Overall, is an effective teacher.}}}

\medskip

\entry{Spring 2020}
			{Theory and Methodology of Political Science}
			{\ucr}
			{This course will lay out the enterprise of empirical research: the structure and content of theories, the formulation of testable hypotheses, and the processes of generating and gathering data. Then discuss how to examine these data, as well as the logic of empirical tests, the evaluation of relationships between two variables, the consideration of competing hypotheses, and the strengths and weaknesses of alternative research designs. Finally, we explore some of the most common statistical tools that political scientists use to answer empirical questions, focusing on how multiple regression analysis and experiments can be used to identify causal relationships and answer questions about the political world.
			\begin{description}
				\item Section 21: \href{https://nrjenkins.github.io/nrjenkins.github.io/files/docs/teaching_evals/POSC114_Spring20_Section21_Scores.pdf}{Median Rating: 7.00/7.00, Mean Rating: 6.57/7.00}, \href{https://nrjenkins.github.io/nrjenkins.github.io/files/docs/teaching_evals/POSC114_Spring20_Section21_Comments.pdf}{Student Feedback}.
				\item Section 22: \href{https://nrjenkins.github.io/nrjenkins.github.io/files/docs/teaching_evals/POSC114_Spring20_Section22_Scores.pdf}{Median Rating: 7.00/7.00, Mean Rating: 6.91/7.00}, \href{https://nrjenkins.github.io/nrjenkins.github.io/files/docs/teaching_evals/POSC114_Spring20_Section22_Comments.pdf}{Student Feedback}.
				\item Section 23: \href{https://nrjenkins.github.io/nrjenkins.github.io/files/docs/teaching_evals/POSC114_Spring20_Section23_Scores.pdf}{Median Rating: 7.00/7.00, Mean Rating: 6.78/7.00}, \href{https://nrjenkins.github.io/nrjenkins.github.io/files/docs/teaching_evals/POSC114_Spring20_Section23_Comments.pdf}{Student Feedback}.
				\item (Mean Department Rating: 6.32, Mean Campus Rating: 6.29)
			\end{description}}

\entry{Winter 2020}
			{Policy Evaluation}
			{\ucr}
			{This class focuses on statistical methods for policy evaluation, that is, the methods one can use to understand the impact of an intervention such as a policy or government program. This course introduces the potential outcomes framework for causal inference and design based causal inference techniques including randomized control trials, regression discontinuity designs, difference-in-difference estimation, and instrumental variables regression.
			\begin{description}
				\item Section 21: \href{https://nrjenkins.github.io/nrjenkins.github.io/files/docs/teaching_evals/PBPL220_Winter20_Scores.pdf}{Median Rating: 7.00/7.00, Mean Rating: 6.84/7.00}, \href{https://nrjenkins.github.io/nrjenkins.github.io/files/docs/teaching_evals/PBPL220_Winter20_Comments.pdf}{Student Feedback}.
				\item (Mean Department Rating: 6.32, Mean Campus Rating: 6.28)
			\end{description}}

\entry{Winter 2019}
			{American Politics}
			{\ucr}
			{An introduction to the principles and practices of government. Focuses on the policy process and selected political issues in the United States.
			\begin{description}
				\item Section 24: \href{https://nrjenkins.github.io/nrjenkins.github.io/files/docs/teaching_evals/POSC010_Winter19_Section24_Scores.pdf}{Median Rating: 7.00/7.00, Mean Rating: 6.75/7.00}, \href{https://nrjenkins.github.io/nrjenkins.github.io/files/docs/teaching_evals/POSC010_Winter19_Section24_Comments.pdf}{Student Feedback},
				\item Section 25: \href{https://nrjenkins.github.io/nrjenkins.github.io/files/docs/teaching_evals/POSC010_Winter19_Section25_Scores.pdf}{Median Rating: 7.00/7.00, Mean Rating: 6.80/7.00}, \href{https://nrjenkins.github.io/nrjenkins.github.io/files/docs/teaching_evals/POSC010_Winter19_Section25_Comments.pdf}{Student Feedback}.
				\item Section 26: \href{https://nrjenkins.github.io/nrjenkins.github.io/files/docs/teaching_evals/POSC010_Winter19_Section26_Scores.pdf}{Median Rating: 7.00/7.00, Mean Rating: 6.54/7.00}, \href{https://nrjenkins.github.io/nrjenkins.github.io/files/docs/teaching_evals/POSC010_Winter19_Section26_Comments.pdf}{Student Feedback}.
				\item (Mean Department Rating: 6.42, Mean Campus Rating: 6.21)
			\end{description}}
			

\entry{Fall 2018}
			{American Politics}
			{\ucr}
			{An introduction to the principles and practices of government. Focuses on the policy process and selected political issues in the United States.
			\begin{description}
				\item Section 22: \href{https://nrjenkins.github.io/nrjenkins.github.io/files/docs/teaching_evals/POSC010_Fall18_Section22_Scores.pdf}{Median Rating: 7.00/7.00, Mean Rating: 6.78/7.00}, \href{https://nrjenkins.github.io/nrjenkins.github.io/files/docs/teaching_evals/POSC010_Fall18_Section22_Comments.pdf}{Student Feedback}.
				\item Section 25: \href{https://nrjenkins.github.io/nrjenkins.github.io/files/docs/teaching_evals/POSC010_Fall18_Section25_Scores.pdf}{Median Rating: 7.00/7.00, Mean Rating: 6.78/7.00}, \href{https://nrjenkins.github.io/nrjenkins.github.io/files/docs/teaching_evals/POSC010_Fall18_Section25_Comments.pdf}{Student Feedback}.
				\item Section 31: \href{https://nrjenkins.github.io/nrjenkins.github.io/files/docs/teaching_evals/POSC010_Fall18_Section31_Scores.pdf}{Median Rating: 6.00/7.00, Mean Rating: 5.80/7.00}, \href{https://nrjenkins.github.io/nrjenkins.github.io/files/docs/teaching_evals/POSC010_Fall18_Section31_Comments.pdf}{Student Feedback}.
				\item (Mean Department Rating: 6.48, Mean Campus Rating: 6.19)
			\end{description}}
			
\entry{Spring 2017}
			{Forecasting Lab}
			{\csulb}
			{Principles and methods of forecasting. Evaluation of the reliability of existing forecasting techniques. Also covers use of the macroeconomic model as a basis for forecasting and the role of forecasts in the formulation of national economic policy.
			\begin{description}
				\item \href{https://nrjenkins.github.io/nrjenkins.github.io/files/docs/teaching_evals/Econ420_Teaching_Evaluations_Spring2017.pdf}{Median Rating: 6.00/6.00, Mean Rating: 5.71/6.00}
			\end{description}}

\entry{Spring 2017}
			{Economic Statistics Lab}
			{\csulb}
			{Use of descriptive and inferential statistical concepts for the analysis of economic data. Topics include measures of central tendency and dispersion, probability theory, discrete and continuous probability distributions, hypothesis testing, regression and correlation analysis.
			\begin{description}
				\item \href{https://nrjenkins.github.io/nrjenkins.github.io/files/docs/teaching_evals/Econ380_Teaching_Evaluations_Spring2017.pdf}{Median Rating: 6.00/6.00, Mean Rating: 5.66/6.00}
			\end{description}}
			
\entry{Spring 2017}
			{Intermediate Macroeconomics}
			{\csulb}
			{Determinants of levels of income, employment, and prices; of secular and cyclical changes in economic activity; and of the effects of public policies upon aggregative economic experience.}
			
\entry{2016}
			{Intermediate Microeconomics}
			{\csulb}
			{Analysis of economic concepts and their applications to business situations. Emphasis on supply and demand analysis, costs of production, variations of competition and monopoly, revenues, prices, profits and losses, and other aspects of the operations of the business enterprise.}

\entry{Fall 2016}
			{Introduction to Econometrics Lab}
			{\csulb}
			{Introduction to econometrics, with a focus on understanding and applying the classical linear regression model. Emphasis placed on applications of regression analysis.
			\begin{description}
				\item \href{https://nrjenkins.github.io/nrjenkins.github.io/files/docs/teaching_evals/Econ485_Teaching_Evaluations_Fall2016.pdf}{Median Rating: 6.00/6.00, Mean Rating: 5.61/6.00}
			\end{description}}

\entry{Fall 2016}
			{Economic Statistics Lab}
			{\csulb}
			{Use of descriptive and inferential statistical concepts for the analysis of economic data. Topics include measures of central tendency and dispersion, probability theory, discrete and continuous probability distributions, hypothesis testing, regression and correlation analysis.
			\begin{description}
				\item \href{https://nrjenkins.github.io/nrjenkins.github.io/files/docs/teaching_evals/Econ380_Teaching_Evaluations_Fall2016.pdf}{Median Rating: 6.00/6.00, Mean Rating: 5.34/6.00} 
			\end{description}}

\entry{Spring 2016}
			{Economic Statistics Lab}
			{\csulb}
			{Use of descriptive and inferential statistical concepts for the analysis of economic data. Topics include measures of central tendency and dispersion, probability theory, discrete and continuous probability distributions, hypothesis testing, regression and correlation analysis.
			\begin{description}
				\item \href{https://nrjenkins.github.io/nrjenkins.github.io/files/docs/teaching_evals/Econ380_Teaching_Evaluations_Spring2016.pdf}{Median Rating: 6.00/6.00, Mean Rating: 5.88/6.00}
			\end{description}}

\entry{Fall 2015}
			{Economic Statistics Lab}
			{\csulb}
			{Use of descriptive and inferential statistical concepts for the analysis of economic data. Topics include measures of central tendency and dispersion, probability theory, discrete and continuous probability distributions, hypothesis testing, regression and correlation analysis.
			\begin{description}
				\item \href{https://nrjenkins.github.io/nrjenkins.github.io/files/docs/teaching_evals/Econ380_Teaching_Evaluations_Fall2015.pdf}{Median Rating: 6.00/6.00, Mean Rating: 5.89/6.00} 
			\end{description}}

\entry{2015 - 2017}
			{Principles of Microeconomics}
			{\csulb}
			{Business organization, price theory, allocation of resources, distribution of income, public economy.}

\vspace{.25cm}


% Additional Training %
\section{Additional Training}

\medskip

\entry{2020}
			{Structural Equation Modeling in Longitudinal Research \href{https://nrjenkins.github.io/files/docs/cv/APA_ATI_Completion_Certificate.pdf}{\cert}}
			{Longitudinal Research Institute}
			{This ATI is designed to highlight recent methodological advances in the analysis of longitudinal psychological data using structural equation modeling (SEM). The training is intended for faculty, postdocs and advanced graduate students who are familiar with SEM (e.g., took an introductory SEM course). The workshop covers a range of topics, including growth models, factorial invariance, dealing with incomplete data, growth mixture models, ordinal outcomes, and latent change score models.}

\entry{2020}
			{Chancellor’s Making Excellence Inclusive Graduate Division Diversity Certification Program}
			{\ucr}
			{The Diversity Certificate Program is a 10-week program that is designed to sustain and strengthen a supportive network of UCR graduate students interested in research, pedagogy, skills development, and learning around issues of diversity, equity, and inclusion.}

\entry{2020}
			{University Teaching Certificate}
			{\ucr}
			{The University of California, Riverside’s Graduate Division created the University Teaching Certificate (UTC) Program as a two-quarter instructional training and certification program for graduate students. This highly competitive program is designed to assist graduate students interested in careers as university-level instructors to develop teaching and lecturing strategies, design a teaching philosophy, and become members of the professional teaching community. The goal of the UTC program is to equip graduate students with the necessary skills so that they can be regarded as both outstanding scholars and teachers in their continued work in academia.}

\entry{2018}
			{Inter-university Consortium for Political and Social Research (ICPSR) Summer Program in Quantitative Methods of Social Research \href{https://nrjenkins.github.io/files/docs/cv/ICPSR_Participation_Verification.pdf}{\cert} \href{https://nrjenkins.github.io/files/docs/cv/ICPSR_Grades.pdf}{\grades}}
			{University of Michigan, Ann Arbor}
			{The \href{https://www.icpsr.umich.edu/icpsrweb/content/sumprog/about.html}{ICPSR Summer Program} provides rigorous, hands-on training in statistical techniques, research methodologies, and data analysis. ICPSR Summer Program courses emphasize the integration of methodological strategies with the theoretical and practical concerns that arise in research on substantive issues. I completed the following courses: 
			\begin{itemize}
				\item Bayesian Modeling for the Social Sciences I
				\item Multilevel Models I: Introduction and Application
				\item Mathematics for Social Scientists III
			\end{itemize}
Please click \href{https://nrjenkins.github.io/files/docs/cv/ICPSR_Participation_Verification.pdf}{here} to view my certificate of completion and click \href{https://nrjenkins.github.io/files/docs/cv/ICPSR_Grades.pdf}{here} to view the grades I received.} 

\entry{2017}
			{Macroeconomic Forecasting \href{https://courses.edx.org/certificates/d1ac1a8d2755457a9651b9dad73c2602}{\cert}}
			{IMF Institute for Capacity Development}
			{A 9-week course focused on developing forecasts and economic models for scenario analysis used in the design and implementation of macroeconomic policy. Certificate available \href{https://courses.edx.org/certificates/d1ac1a8d2755457a9651b9dad73c2602}{here}.} 

\entry{2015}
			{Plotting in R}
			{American Statistical Association, Orange Country-Log Beach Chapter}
			{A workshop focused on a data centric introduction to using R, in a reproducible way, incorporating lots of data graphics and exploratory data analysis.}

\vspace{.25cm}


% Coursework %
\section{Relevant Coursework}

\medskip

\entry{Politics}
			{Comparative Political Economy, Political Economy of International Trade, Political Economy of International Finance, Political Economy of International Migration, American Bureaucratic Institutions, Representation, US Congress, Politics of Race Immigration and Ethnicity, Public Opinion and Mass Media, American Electoral Behavior, Comparative Political Behavior}
			{}
			
\vspace{0.10cm}
			
\entry{Economics}
			{Advanced Macroeconomics, Advanced Microeconomics, International Finance, Development Economics}
			{}

\vspace{0.10cm}
			
\entry{Methods}
			{Econometrics I, Econometrics II, Maximum Likelihood Estimation, Multilevel Modeling, Bayesian Modeling for the Social Sciences, Macroeconomic Forecasting, Qualitative Research Methods}
			{}

\vspace{0.10cm}
			
\entry{Math}
			{Calculus I, Calculus II, Calculus III, Linear Algebra, Mathematics for Social Scientists III}
			{}


% Software and Programming %
\section{Software and Programming Knowledge}

\medskip

\entry{Languages}
			{R, Stan, JAGS, BUGS, \LaTeX, HTML}
			{}
		
\entry{Programs}
			{RStudio, STATA, EViews}
			{}
			

\end{document}